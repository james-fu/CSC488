% ----------------------------------------------------------------
% AMS-LaTeX Paper ************************************************
% **** -----------------------------------------------------------
\documentclass[oneside]{amsart}
\usepackage{graphicx}
\usepackage{color}
\usepackage[letterpaper]{geometry}
\usepackage[colorlinks=false,
            pdfborder={0 0 0},
            pdftitle={CSC488 A2},
            pdfauthor={Daniel Bloemendal},
            pdfsubject={CSC488},
            pdfstartview=FitH,
            pdfmenubar=false,
            pdfdisplaydoctitle=true,
            bookmarks=false]{hyperref}
\usepackage{subcaption}
\usepackage{mathtools}
\usepackage{listings}
\usepackage[table]{xcolor}
% ----------------------------------------------------------------
\vfuzz2pt % Don't report over-full v-boxes if over-edge is small
\hfuzz2pt % Don't report over-full h-boxes if over-edge is small
% THEOREMS -------------------------------------------------------
\newtheorem{thm}{Theorem}[section]
\newtheorem{cor}[thm]{Corollary}
\newtheorem{lem}[thm]{Lemma}
\newtheorem{prop}[thm]{Proposition}
\theoremstyle{definition}
\newtheorem{defn}[thm]{Definition}
\theoremstyle{remark}
\newtheorem{rem}[thm]{Remark}
\numberwithin{equation}{section}
% MATH -----------------------------------------------------------
\newcommand{\norm}[1]{\left\Vert#1\right\Vert}
\newcommand{\abs}[1]{\left\vert#1\right\vert}
\newcommand{\set}[1]{\left\{#1\right\}}
\newcommand{\Real}{\mathbb R}
\newcommand{\eps}{\varepsilon}
\newcommand{\To}{\longrightarrow}
\newcommand{\BX}{\mathbf{B}(X)}
\newcommand{\A}{\mathcal{A}}
\newcommand{\e}{\mathrm{e}}
\newcommand{\AND}{\wedge}
\newcommand{\OR}{\vee}
\newcommand{\NOT}{\neg}
\newcommand{\IMPLIES}{\to}
\newcommand{\TRUE}{\top}
\newcommand{\FALSE}{\bot}
\newcommand{\EQUALS}{\equiv}
\DeclareMathOperator{\sech}{sech}
\newcolumntype{B}{>{\columncolor{black}\color{white}}c}
% ----------------------------------------------------------------

\begin{document}

\title[CSC488 A2]{CSC488\\ASSIGNMENT 2\\Grammar Design}
\author{Daniel Bloemendal}

% ----------------------------------------------------------------
\begin{titlepage}
\maketitle
\thispagestyle{empty}
\tableofcontents
\end{titlepage}
% ----------------------------------------------------------------

\section{Overview}
We began by translating the production rules in the reference grammar into Java CUP syntax. However,
the reference grammar is certainly not LALR. Furthermore, the reference grammar does not observe any
of the operator precedence rules described in the language documentation. So the next goal was to
begin transforming the production rules into LALR friendly rules, observing precedence in
expressions. The challenges encountered during this transformation will be outlined below.

\section{Shift/Reduce conflicts}
The reference grammar uses the following style to accept lists. It is used for instance to recognize
lists of statements or declarations.
\begin{lstlisting}
statement
    ::= ...
    |   ...
    |   statement statement
    ;
\end{lstlisting}
Unfortunately with LALR parsers, as soon as a statement is shifted on to the stack, there are two
valid actions. The statement on stack can be reduced to the ``statement'' non-terminal, or an
additional statement can be shifted on to the stack to satisfy the alternative
\begin{lstlisting}
    |   statement statement
\end{lstlisting}
The solution was to add explicit list production rules as follows.
\begin{lstlisting}
statement_list
    ::= statement_list statement
    |   statement
    ;
\end{lstlisting}
This resolves the issue as a reduction to ``statement\_list'' is only possible if there are no
remaining statements left to shift on to the stack. This transformation was applied for statements,
declarations and other such lists.

\section{Reduce/Reduce conflicts}
\begin{lstlisting}
variable
    ::= variablename
    |   parametername
    |   variablename L_SQUARE expression R_SQUARE
    |   variablename L_SQUARE expression COMMA expression R_SQUARE
    ;
\end{lstlisting}

% ----------------------------------------------------------------
\end{document}
% ----------------------------------------------------------------
