% ----------------------------------------------------------------
% AMS-LaTeX Paper ************************************************
% **** -----------------------------------------------------------
\documentclass[oneside]{amsart}
\usepackage{graphicx}
\usepackage{color}
\usepackage[letterpaper]{geometry}
\usepackage[colorlinks=false,
            pdfborder={0 0 0},
            pdftitle={CSC488 A5},
            pdfauthor={Daniel Bloemendal},
            pdfsubject={CSC488},
            pdfstartview=FitH,
            pdfmenubar=false,
            pdfdisplaydoctitle=true,
            bookmarks=false]{hyperref}
\usepackage{subcaption}
\usepackage{mathtools}
\usepackage{listings}
\usepackage[table]{xcolor}
\usepackage{syntax}
% ----------------------------------------------------------------
\vfuzz2pt % Don't report over-full v-boxes if over-edge is small
\hfuzz2pt % Don't report over-full h-boxes if over-edge is small
% THEOREMS -------------------------------------------------------
\newtheorem{thm}{Theorem}[section]
\newtheorem{cor}[thm]{Corollary}
\newtheorem{lem}[thm]{Lemma}
\newtheorem{prop}[thm]{Proposition}
\theoremstyle{definition}
\newtheorem{defn}[thm]{Definition}
\theoremstyle{remark}
\newtheorem{rem}[thm]{Remark}
\numberwithin{equation}{section}
% MATH -----------------------------------------------------------
\newcommand{\norm}[1]{\left\Vert#1\right\Vert}
\newcommand{\abs}[1]{\left\vert#1\right\vert}
\newcommand{\set}[1]{\left\{#1\right\}}
\newcommand{\Real}{\mathbb R}
\newcommand{\eps}{\varepsilon}
\newcommand{\To}{\longrightarrow}
\newcommand{\BX}{\mathbf{B}(X)}
\newcommand{\A}{\mathcal{A}}
\newcommand{\e}{\mathrm{e}}
\newcommand{\AND}{\wedge}
\newcommand{\OR}{\vee}
\newcommand{\NOT}{\neg}
\newcommand{\IMPLIES}{\to}
\newcommand{\TRUE}{\top}
\newcommand{\FALSE}{\bot}
\newcommand{\EQUALS}{\equiv}
\DeclareMathOperator{\sech}{sech}
\newcolumntype{B}{>{\columncolor{black}\color{white}}c}
% ----------------------------------------------------------------
\lstset {
    basicstyle=\fontsize{8}{11}\selectfont\ttfamily,
    frame=none,
    numbers=none
}
% ----------------------------------------------------------------

\begin{document}

\title[CSC488 A5]{CSC488\\ASSIGNMENT 5\\Assembler}
\author{Daniel Bloemendal}

% ----------------------------------------------------------------
\begin{titlepage}
\maketitle
\thispagestyle{empty}
\tableofcontents
\end{titlepage}
% ----------------------------------------------------------------

\section{Overview}
The first goal, before beginning work on code generation, was to implement an assembler for our IR
instruction set. This included instructions with label operands, instead of static addresses, and
each macro, or IR instruction, outlined in the code generation templates. The entire IR instruction
set will be covered again in detail in this document. There are also a few notable bugs that were
fixed since the code generation templates in A4.

The assembler was designed to be decoupled from the code generator. It is standalone and can even be
executed from command line on IR assembly programs. A run script is provided along with an
``\texttt{ant}'' command to build the assembler. To build the assembler, run
``\texttt{ant assembler}''. To execute the assembler from command line, run
``\texttt{./RUNASSEMBLER.sh <source.ir>}''

We will begin by outlining the grammar of the assembler. We will then explore the label system and
IR instructions.

\section{Grammar}
\noindent The following is the grammar for assembly programs in EBNF. \\
Please refer to http://en.wikipedia.org/wiki/Extended_Backus-Naur_Form for more information. \\

\setlength{\grammarindent}{12em}
\begin{grammar}
<program>           ::= \{ <section> \}

<section>           ::= `SECTION' <label> `\\n'
\alt                  \{ <instruction> \}

<instruction>       ::= ( <machine_operation> | <ir_operation> )  \{ <operand> \} `\\n'

<machine_operation> ::= `HALT' | `ADDR' | `LOAD' | `STORE' | `PUSH' | `PUSHMT' | `SETD' | `POP' | `POPN' | `DUP' | `DUPN' | `BR' | `BF' | `NEG' | `ADD' | `SUB' | `MUL' | `DIV' | `EQ' | `LT' | `OR' | `SWAP' | `READC' | `PRINTC' | `READI' | `PRINTI' | `TRON' | `TROFF' | `ILIMIT'

<ir_operation>      ::= `PUSHSTR' | `SETUPCALL' | `JMP' | `BFALSE' | `NOT' | `SAVECTX' | `RESTORECTX' | `RESERVE'

<operand>           ::= <integer_operand>
\alt                    <boolean_operand>
\alt                    <label_operand>
\alt                    <string_operand>

<integer_operand>   ::= [`-'], <digit>, \{<digit>\}

<boolean_operand>   ::= `\$true' | `\$false'

<label_operand>     ::= <label>

<string_operand>    ::= `"' \{ <character> \} `"'

<label>             ::= <letter>, \{<letter> | <digit>\}

<letter>            ::= `a'..`z' | `A'..`Z' | `_'

<digit>             ::= `0'..`9'
\end{grammar}

\section{}


% ----------------------------------------------------------------
\end{document}
% ----------------------------------------------------------------
