% ----------------------------------------------------------------
% AMS-LaTeX Paper ************************************************
% **** -----------------------------------------------------------
\documentclass[oneside]{amsart}
\usepackage{graphicx}
\usepackage{color}
\usepackage[letterpaper]{geometry}
\usepackage[colorlinks=false,
            pdfborder={0 0 0},
            pdftitle={CSC488 A5},
            pdfauthor={Daniel Bloemendal},
            pdfsubject={CSC488},
            pdfstartview=FitH,
            pdfmenubar=false,
            pdfdisplaydoctitle=true,
            bookmarks=false]{hyperref}
\usepackage{subcaption}
\usepackage{mathtools}
\usepackage{listings}
\usepackage[table]{xcolor}
\usepackage{syntax}
% ----------------------------------------------------------------
\vfuzz2pt % Don't report over-full v-boxes if over-edge is small
\hfuzz2pt % Don't report over-full h-boxes if over-edge is small
% THEOREMS -------------------------------------------------------
\newtheorem{thm}{Theorem}[section]
\newtheorem{cor}[thm]{Corollary}
\newtheorem{lem}[thm]{Lemma}
\newtheorem{prop}[thm]{Proposition}
\theoremstyle{definition}
\newtheorem{defn}[thm]{Definition}
\theoremstyle{remark}
\newtheorem{rem}[thm]{Remark}
\numberwithin{equation}{section}
% MATH -----------------------------------------------------------
\newcommand{\norm}[1]{\left\Vert#1\right\Vert}
\newcommand{\abs}[1]{\left\vert#1\right\vert}
\newcommand{\set}[1]{\left\{#1\right\}}
\newcommand{\Real}{\mathbb R}
\newcommand{\eps}{\varepsilon}
\newcommand{\To}{\longrightarrow}
\newcommand{\BX}{\mathbf{B}(X)}
\newcommand{\A}{\mathcal{A}}
\newcommand{\e}{\mathrm{e}}
\newcommand{\AND}{\wedge}
\newcommand{\OR}{\vee}
\newcommand{\NOT}{\neg}
\newcommand{\IMPLIES}{\to}
\newcommand{\TRUE}{\top}
\newcommand{\FALSE}{\bot}
\newcommand{\EQUALS}{\equiv}
\DeclareMathOperator{\sech}{sech}
\newcolumntype{B}{>{\columncolor{black}\color{white}}c}
% ----------------------------------------------------------------
\lstset {
    basicstyle=\fontsize{8}{11}\selectfont\ttfamily,
    frame=none,
    numbers=none,
    escapeinside={<@}{@>}
}
% ----------------------------------------------------------------

\begin{document}

\title[CSC488 A5]{CSC488\\ASSIGNMENT 5\\Assembler}
\author{Daniel Bloemendal}

% ----------------------------------------------------------------
\begin{titlepage}
\maketitle
\thispagestyle{empty}
\tableofcontents
\end{titlepage}
% ----------------------------------------------------------------

\section{Overview}
The first goal, before beginning work on code generation, was to implement an assembler for our IR
instruction set. This included instructions with label operands, instead of static addresses, and
each macro, or IR instruction, outlined in the code generation templates. The entire IR instruction
set will be covered again in detail in this document. There are also a few notable bugs that were
fixed since the code generation templates in A4.

The assembler was designed to be decoupled from the code generator. It is standalone and can even be
executed from command line on IR assembly programs. A run script is provided along with an
``\texttt{ant}'' command to build the assembler. To build the assembler, run
``\texttt{ant assembler}''. To execute the assembler from command line, run
``\texttt{./RUNASSEMBLER.sh <source.ir>}''

\section{Grammar}
\noindent The following is the grammar for IR assembly programs in EBNF. \\
Please refer to http://en.wikipedia.org/wiki/Extended_Backus-Naur_Form for more information. \\

\setlength{\grammarindent}{12em}
\begin{grammar}
<program>           ::= \{ <section> \}

<section>           ::= `SECTION', `.', <ident>, `\\n', \\
                        \{ [<label>], [<instruction>], `\\n' \}

<label>             ::= <ident>, `:'

<instruction>       ::= ( <machine_operation> | <ir_operation> ), \{ <operand> \}

<machine_operation> ::= `HALT' | `ADDR' | `LOAD' | `STORE' | `PUSH' | `PUSHMT' | `SETD' | `POP' | `POPN' | `DUP' | `DUPN' | `BR' | `BF' | `NEG' | `ADD' | `SUB' | `MUL' | `DIV' | `EQ' | `LT' | `OR' | `SWAP' | `READC' | `PRINTC' | `READI' | `PRINTI' | `TRON' | `TROFF' | `ILIMIT'

<ir_operation>      ::= `PUSHSTR' | `SETUPCALL' | `JMP' | `BFALSE' | `NOT' | `SAVECTX' | `RESTORECTX' | `RESERVE'

<operand>           ::= <value_operand>
\alt                    <boolean_operand>
\alt                    <string_operand>

<value_operand>     ::= <integer_operand>
\alt                    <label_operand>

<integer_operand>   ::= [`-'], <digit>, \{<digit>\}

<boolean_operand>   ::= `\$true' | `\$false'

<label_operand>     ::= <ident>

<string_operand>    ::= `"', \{ ? all characters ? \}, `"'

<ident>             ::= <letter>, \{<letter> | <digit>\}

<letter>            ::= `a'..`z' | `A'..`Z' | `_'

<digit>             ::= `0'..`9'
\end{grammar}

\section{Labels}
The biggest difference between the IR assembly language and the assembly language built into
\texttt{Machine} is the support for labels. Labels can be thought of as instructions of with zero
size and no operation. The addresses of labels are stored as instructions are assembled. Afterwards,
any label operand pointing to a valid label is set to resolve to the computed address of the label
it points to.

\section{Instructions}
In addition to support for labels, the assembler also provides an extended instruction set. The
following instructions outlined below are not available in \texttt{Machine}.
\subsection{\texttt{PUSHSTR}(\texttt{$\langle$string_operand$\rangle$} \texttt{string})}
This instruction adds \texttt{string} to the text constant pool, if it has not already been
added, and obtains the address of the string in the pool. The address is then pushed on to the
stack.
\begin{lstlisting}
    PUSH ``address of string in constant pool''
\end{lstlisting}
\subsection{\texttt{SETUPCALL}(\texttt{$\langle$value_operand$\rangle$} \texttt{return_address})}
\texttt{SETUPCALL} sets up a call, according to calling convention designed for the compiler, by
pushing a place holder for the return value, and by pushing \texttt{return_address}, the return
address for the call. The placeholder for the return value is set to be \texttt{Machine.UNDEFINED},
to support returning from a function or procedure without an explicit ``\texttt{return}'' or
``\texttt{result}'' statement. For a more detailed discussion on the calling convention, see the
code generation documentation.
\begin{lstlisting}
    PUSH UNDEFINED
    PUSH return_address
\end{lstlisting}
\subsection{\texttt{JMP}(\texttt{$\langle$value_operand$\rangle$} \texttt{address})}
\texttt{JMP} pushes $\texttt{address}$ on to the stack and then performs a branch.
\begin{lstlisting}
    PUSH address
    BR
\end{lstlisting}
\subsection{\texttt{BFALSE}(\texttt{$\langle$value_operand$\rangle$} \texttt{address})}
Like \texttt{JMP}, \texttt{BFALSE} is also a short hand for pushing an address and then branching.
This time, the branch is the conditional \texttt{BF}.
\begin{lstlisting}
    PUSH address
    BF
\end{lstlisting}
\subsection{\texttt{NOT}}
Unfortunately the machine has no built in method of negating boolean values. \texttt{NOT} provides
this much needed functionality. It does so by comparing the top of the stack with \texttt{\$false}.
To see this, suppose $top = \texttt{\$true}$, then $(\texttt{\$true} = \texttt{\$false}) =
\texttt{\$false}$ on the other hand if $top = \texttt{\$false}$ then $(\texttt{\$false} =
\texttt{\$false}) = \texttt{\$true}$
\begin{lstlisting}
    PUSH $false
    EQ
\end{lstlisting}

\newpage

\subsection{\texttt{SAVECTX}(\texttt{$\langle$value_operand$\rangle$} \texttt{lexical_level})}
\texttt{SAVECTX} is used at the beginning of a major scope to preserve any previous display set by
a major scope at the same lexical level, a sibling scope. After preserving the display, it sets the
display to the current address stored in the stack pointer. This is part of the calling convention
and is explained in more detail in the code generation documentation.
\begin{lstlisting}
    ADDR lexical_level 0
    PUSHMT
    SETD lexical_level
\end{lstlisting}
\subsection{\texttt{RESTORECTX}(\texttt{$\langle$value_operand$\rangle$} \texttt{lexical_level},
\texttt{$\langle$value_operand$\rangle$} \texttt{argument_count})} \ \\
\texttt{RESTORECTX} is used at the end of major scope restore the display saved by \texttt{SAVECTX}.
It also pops \texttt{argument_count} arguments off of the stack. Again, for more details on the
calling convention, refer to the code generation documentation.
\begin{lstlisting}
    SETD lexical_level
    PUSH argument_count
    POPN
\end{lstlisting}
\subsection{\texttt{RESERVE}(\texttt{$\langle$value_operand$\rangle$} \texttt{words})}
\texttt{RESERVE} is used to reserve \texttt{words} of memory on the stack. It is current used in the
prolog of a major scope to reserve memory for locals.
\begin{lstlisting}
    PUSH 0
    PUSH words
    DUPN
\end{lstlisting}

\section{Example}
The example below is the currently used runtime library for the code generator. It contains only
one function right now, \texttt{print}. It serves as a good demonstration of the IR assembly, using
a number of the different IR instructions.
\begin{lstlisting}[numbers=left]
; ------------------------------------
; Start of runtime library
; ------------------------------------
    SECTION <@\textcolor{red}{.library}@>

<@\textcolor{red}{print:}@>
    SAVECTX 0
    PUSH 0
<@\textcolor{red}{__print_start:}@>
    DUP
    ADDR 0 -2
    LOAD
    LOAD
    LT
    PUSH <@\textcolor{red}{__print_end}@>
    BF
    DUP
    ADDR 0 -2
    LOAD
    PUSH 1
    ADD
    ADD
    LOAD
    PRINTC
    PUSH 1
    ADD
    PUSH <@\textcolor{red}{__print_start}@>
    BR
<@\textcolor{red}{__print_end:}@> POP
    RESTORECTX 0 1
    BR
; ------------------------------------
; End of runtime library
; ------------------------------------
\end{lstlisting}

\section{Error handling}
The machine has a number of nontrivial limitations that are easy to run into. The limited memory
and 16 bit operands ensure that it is easy to hit the limits. The assembler is careful to perform
bounds checking on integer operands, enforcing the \texttt{Machine.MIN_INTEGER} and
\texttt{Machine.MAX_INTEGER} bounds. In addition to integer bounds checking, the assembler also will
gracefully handle programs that exceed the available machine memory \texttt{Machine.maxMemory}. An
exception will be thrown when these errors are encountered.

\section{Changes}
Since assignment 4 bugs were identified and fixed. First, \texttt{RESTORECTX} incorrectly restored
the display. Instead of loading the saved display via \texttt{ADDR}, and then popping it off the
stack afterwards, it made more sense to simply consume it and restore the display simultaneously
with \texttt{SETD}. Other issues included the correct offsets in a stack frame. The stack frame
and calling conventions are discussed in detail in the code generation documentation.

\newpage

\section{Credits}
\begin{center}
\begin{tabular}{ll}
    src/compiler488/codegen/assembler/Operand.java & Daniel Bloemendal \\
    src/compiler488/codegen/assembler/Section.java & Daniel Bloemendal \\
    src/compiler488/codegen/assembler/StringOperand.java & Daniel Bloemendal \\
    src/compiler488/codegen/assembler/Instruction.java & Daniel Bloemendal \\
    src/compiler488/codegen/assembler/Processor.java & Daniel Bloemendal \\
    src/compiler488/codegen/assembler/Assembler.java & Daniel Bloemendal \\
    src/compiler488/codegen/assembler/InvalidInstructionException.java & Daniel Bloemendal \\
    src/compiler488/codegen/assembler/IntegerOperand.java & Daniel Bloemendal \\
    src/compiler488/codegen/assembler/ProgramSizeException.java & Daniel Bloemendal \\
    src/compiler488/codegen/assembler/LabelNotResolvedException.java & Daniel Bloemendal \\
    src/compiler488/codegen/assembler/LabelOperand.java & Daniel Bloemendal \\
    src/compiler488/codegen/assembler/ir/AssemblerMachineEmitter.java & Mike Qin \\
    src/compiler488/codegen/assembler/ir/AssemblerIREmitter.java & Mike Qin \\
    src/compiler488/codegen/assembler/ir/Emitter.java & Mike Qin \\
\end{tabular}
\end{center}

% ----------------------------------------------------------------
\end{document}
% ----------------------------------------------------------------
