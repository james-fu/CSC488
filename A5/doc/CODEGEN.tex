% ----------------------------------------------------------------
% AMS-LaTeX Paper ************************************************
% **** -----------------------------------------------------------
\documentclass[oneside]{amsart}
\usepackage{graphicx}
\usepackage{color}
\usepackage[letterpaper]{geometry}
\usepackage[colorlinks=false,
            pdfborder={0 0 0},
            pdftitle={CSC488 A5},
            pdfauthor={Daniel Bloemendal},
            pdfsubject={CSC488},
            pdfstartview=FitH,
            pdfmenubar=false,
            pdfdisplaydoctitle=true,
            bookmarks=false]{hyperref}
\usepackage{caption}
\usepackage{subcaption}
\usepackage{mathtools}
\usepackage{listings}
\usepackage[table]{xcolor}
% ----------------------------------------------------------------
\vfuzz2pt % Don't report over-full v-boxes if over-edge is small
\hfuzz2pt % Don't report over-full h-boxes if over-edge is small
% THEOREMS -------------------------------------------------------
\newtheorem{thm}{Theorem}[section]
\newtheorem{cor}[thm]{Corollary}
\newtheorem{lem}[thm]{Lemma}
\newtheorem{prop}[thm]{Proposition}
\theoremstyle{definition}
\newtheorem{defn}[thm]{Definition}
\theoremstyle{remark}
\newtheorem{rem}[thm]{Remark}
\numberwithin{equation}{section}
% MATH -----------------------------------------------------------
\newcommand{\norm}[1]{\left\Vert#1\right\Vert}
\newcommand{\abs}[1]{\left\vert#1\right\vert}
\newcommand{\set}[1]{\left\{#1\right\}}
\newcommand{\Real}{\mathbb R}
\newcommand{\eps}{\varepsilon}
\newcommand{\To}{\longrightarrow}
\newcommand{\BX}{\mathbf{B}(X)}
\newcommand{\A}{\mathcal{A}}
\newcommand{\e}{\mathrm{e}}
\newcommand{\AND}{\wedge}
\newcommand{\OR}{\vee}
\newcommand{\NOT}{\neg}
\newcommand{\IMPLIES}{\to}
\newcommand{\TRUE}{\top}
\newcommand{\FALSE}{\bot}
\newcommand{\EQUALS}{\equiv}
\DeclareMathOperator{\sech}{sech}
\newcolumntype{B}{>{\columncolor{black}\color{white}}c}
% ----------------------------------------------------------------
\lstset {
    frame=top,
    frame=bottom,
    numbers=none,
    xleftmargin=14pt,
    framexleftmargin=17pt,
    framexrightmargin=17pt,
    framexbottommargin=2pt,
    framextopmargin=2pt   
}
\captionsetup[lstlisting]{singlelinecheck=false, margin=0pt, font={sf},labelsep=space,labelfont=bf}
% ----------------------------------------------------------------

\begin{document}

\title[CSC488 A5]{CSC488\\ASSIGNMENT 5\\Code Generator}
\author{Daniel Bloemendal \\ Oren Watson}

% ----------------------------------------------------------------
\begin{titlepage}
\maketitle
\thispagestyle{empty}
\tableofcontents
\end{titlepage}
% ----------------------------------------------------------------

\section{Instructions}
\subsection{Bounds checking}
\subsection{Code dump \& syntax highlighting}

\section{Design}
\subsection{Overview}
The overarching theme in the design of the code generator was to avoid exposing the code generator
class \texttt{CodeGen} to the complexities and finer details of the underlying machine. To that
end, an assembler was developed that hides the complexity of addressing code via a label system and
provides an enhanced instruction set, simplifying the emitted code in \texttt{CodeGen}. It should be
noted that the assembler is also entirely decoupled from the rest of the code generator and stands
on its own. The assembler is covered in more detail in \texttt{doc/ASSEMBLER.pdf}. We made a distinction 
between `major' and `minor' scopes. A major scope is a whole program, a function, or a procedure,
requiring its own stack frame; all other scopes are considered minor scopes. The complexities of 
managing major scopes, their displays, and ensuring that minor scopes are merged into their enclosing 
major scopes, are dealt with by the \texttt{Frame} and \texttt{Table} classes.

\subsection{Coalescing minor scopes}
One of the more involved parts of developing the code generator was to ensure that minor scopes were
coalesced into the enclosing major scope. This is rather important as it avoids draining the rather
limited resource of display registers, 16 in total. In addition, it reduces the size of the
generated code as it avoids having to generate a prolog and epilog for every minor scope. The main
class responsible for managing this is \texttt{Frame}. It lays out the combined locals of all minor
scopes in a given major scope. It does so via an algorithm identical to the record layout algorithm
discussed in class. A frame can be thought of as a record corresponding to a major scope where all
minor scopes are sub-records. One important detail is that sibling minor scopes can be thought of as
belonging to a union. The reason this is a reasonable thing to do is because only one sibling minor
scope is alive at any one time. Therefore, it is safe to overwrite the locals used by any previously
executed sibling minor scope.

We will now proceed to an example. We will define a small program and lay out the locals according
to the algorithm in \texttt{Frame}.
\begin{lstlisting}[caption=Major scope in 488]
{
    var a : integer a := 0
    { var b, c : integer b := 0 c := 0 }
    { var d, e, f : integer d := 0 e := 0 f := 0 }
    { var g : integer g := 0
        { var h : integer h := 0 }
        { var i, j : integer i := 0 j := 0 }
    } 
}
\end{lstlisting}

\newpage

\noindent We will proceed to lay out the locals. We will express our layout using a C structure.
\begin{lstlisting}[caption=Layout as C structure, language=C, directivestyle={\color{black}} emph={short}, emphstyle={\color{blue}}]
struct major_1 {
    short a;
    union {
        struct { short b, c; } minor_1;
        struct { short d, e, f; } minor_2;
        struct {
            short g;
            union {
                struct { short h; } minor_4;
                struct { short i, j; } minor_5;
            } _siblings;
        } minor_3;
    } _siblings;
};
\end{lstlisting}

\subsection{Calling convention}
A key issue that we debated at length was the method by which function and procedure calls would be
implemented. We decided that the most important thing in functions was for the calling code  to have
the result value at the top of the stack when it returns, with no cleanup, so that the result could
immediately be worked with. Therefore, the function cleans up its own arguments and places its
result into a reserved place at the bottom of the stack frame. We also decided that to avoid undue 
complexity, we woudl treat procedures identically to functions, even if this resulted in wasting 
space with an unused result value space.

To call a function, the calling code reserves a place for the result, pushes the return address and
arguments, and then jumps to the function code. The function code then saves the display pointer for
its level, sets the display, and allocates space for its locals. When returning, the function 
deallocates its locals, restores the display pointer to its previous state, pops its arguments
off the stack, and finally returns, leaving the result as the only thing left. The frame therefore has the
following structure, shown in figure \ref{fig:frame}, with the stack growing down. It should be noted
that the base address $D[LL]$ refers to the display set during the prolog of a function via the
\texttt{SAVECTX LL} intermediate instruction. Here, $LL$ refers to the lexical level of the major
scope.

\begin{figure}[h]
\caption{Stack frame}
\begin{tabular}{l|l}
Result & $D[LL] - N - 3$ \\
Return address & $D[LL] - N - 2$  \\
Argument $1$ & $D[LL] - N - 1$  \\
... & ... \\
Argument $N$ & $D[LL] - 2$ \\
Previous display & $D[LL] - 1$ \\
Locals & $D[LL]$ \\
\end{tabular}
\label{fig:frame}
\end{figure}

% ----------------------------------------------------------------
\end{document}
% ----------------------------------------------------------------
