% ----------------------------------------------------------------
% AMS-LaTeX Paper ************************************************
% **** -----------------------------------------------------------
\documentclass[oneside]{amsart}
\usepackage{graphicx}
\usepackage{color}
\usepackage[letterpaper]{geometry}
\usepackage[colorlinks=false,
            pdfborder={0 0 0},
            pdftitle={CSC488 A5},
            pdfauthor={Daniel Bloemendal},
            pdfsubject={CSC488},
            pdfstartview=FitH,
            pdfmenubar=false,
            pdfdisplaydoctitle=true,
            bookmarks=false]{hyperref}
\usepackage{subcaption}
\usepackage{mathtools}
\usepackage{listings}
\usepackage[table]{xcolor}
% ----------------------------------------------------------------
\vfuzz2pt % Don't report over-full v-boxes if over-edge is small
\hfuzz2pt % Don't report over-full h-boxes if over-edge is small
% THEOREMS -------------------------------------------------------
\newtheorem{thm}{Theorem}[section]
\newtheorem{cor}[thm]{Corollary}
\newtheorem{lem}[thm]{Lemma}
\newtheorem{prop}[thm]{Proposition}
\theoremstyle{definition}
\newtheorem{defn}[thm]{Definition}
\theoremstyle{remark}
\newtheorem{rem}[thm]{Remark}
\numberwithin{equation}{section}
% MATH -----------------------------------------------------------
\newcommand{\norm}[1]{\left\Vert#1\right\Vert}
\newcommand{\abs}[1]{\left\vert#1\right\vert}
\newcommand{\set}[1]{\left\{#1\right\}}
\newcommand{\Real}{\mathbb R}
\newcommand{\eps}{\varepsilon}
\newcommand{\To}{\longrightarrow}
\newcommand{\BX}{\mathbf{B}(X)}
\newcommand{\A}{\mathcal{A}}
\newcommand{\e}{\mathrm{e}}
\newcommand{\AND}{\wedge}
\newcommand{\OR}{\vee}
\newcommand{\NOT}{\neg}
\newcommand{\IMPLIES}{\to}
\newcommand{\TRUE}{\top}
\newcommand{\FALSE}{\bot}
\newcommand{\EQUALS}{\equiv}
\DeclareMathOperator{\sech}{sech}
\newcolumntype{B}{>{\columncolor{black}\color{white}}c}
% ----------------------------------------------------------------
\lstset {
    basicstyle=\fontsize{8}{11}\selectfont\ttfamily,
    frame=none,
    numbers=none
}
% ----------------------------------------------------------------

\begin{document}

\title[CSC488 A5]{CSC488\\ASSIGNMENT 5\\Testing}
\author{Daniel Bloemendal}

% ----------------------------------------------------------------
\begin{titlepage}
\maketitle
\thispagestyle{empty}
\tableofcontents
\end{titlepage}
% ----------------------------------------------------------------

\section{Overview}
For code generation our testing methodology broke into two categories, the methodology for passing
tests and the methodology for failing tests.

\section{Passing tests}
For passing tests we were looking for correct execution of programs. The only way to establish that
other than monitoring the machine state during execution is to perform input/output operations. If
you think of a program as computing a Turing computable function, then the way to test whether or
not the program is indeed computing the function is to ensure that for every input it achieves the
expected output. We enhanced the test runner from assignment 3 to with two new commands in the
comment meta language, ``\texttt{@input=string}'' and ``\texttt{@output=<string>}''. The test runner
feeds tested programs the inputs contained in the \texttt{@input} comment commands sequentially via
\texttt{stdin} and then looks for the expected output contained in the \texttt{@output} commands via
\texttt{stdout}.

The passing tests themselves focused on exercising as many language features as possible. For
instance, one of our original sample programs A1a.488 used all of the operators available in
expressions. To test for correctness we needed only annotate the programs with the output meta
commands looking for the expected output. In addition to basic tests of language features we also
included a number of more complicated algorithms to stress all aspects of code generation. We also
included tests for specific edge cases such as deeply nested minor scopes, to ensure that our
coalescence of minor scopes was fully functional.

\section{Failing tests}
For the failing tests, there were three categories. Tests we expected to fail during code
generation, prefixed by \texttt{codegen\_}, tests we expected to fail at assembly, prefixed by
\texttt{assembler\_} and tests we expected to fail at runtime, prefixed by \texttt{runtime\_}. For the
code generator and assembler errors, we expected graceful handling of the errors. These  were all
errors related to machine limitations. For instance one failing test declares a very large scope,
lots of large arrays, which hits the available stack space limit.

\newpage
\newgeometry{left=0.35in,right=0.35in}

\section{Tests}
\rowcolors{2}{gray!25}{white}
\begin{tabular}{l|l|p{8.5cm}}
\rowcolor{black!80}\textcolor{white}{\textbf{Test}} & \textcolor{white}{\textbf{Author}} & \textcolor{white}{\textbf{Description}} \\
\hline
\texttt{pass/A1a.io.488}                       & Daniel Bloemendal & Tests all logical, comparison, and arithmetic operators. \\
\texttt{pass/A1b.io.488}                       & Daniel Bloemendal & Tests using arrays including both forms of array declaration, positive and negative bounds \\
\texttt{pass/A1c.io.488}                       & Oren Watson       & Tests all forms of loop building and loop exit constructs \\
\texttt{pass/ackermann.488}                    & Oren Watson       & Computes the Ackermann function \\
\texttt{pass/bsort.488}                        & Simon Scott       & An implementation of bubble sort \\
\texttt{pass/cash\_register.488}               & Oren Watson       & Performs some functions of a cash register, including computing subtotal, taxes and total \\
\texttt{pass/collatz.488}                      & Oren Watson       & Verifies Collatz conjecture for input numbers \\
\texttt{pass/dectohex.488}                     & Oren Watson       & Converts a decimal number into hexadecimal \\
\texttt{pass/deeply\_nested.488}               & Oren Watson       & A program with deeply nested minor scopes \\
\texttt{pass/euclid.488}                       & Simon Scott       & An implementation of the euclidean algorithm \\
\texttt{pass/four\_fn\_calc.488}               & Oren Watson       & Four function calculator \\
\texttt{pass/multi\_level\_assignment.488}     & Oren Watson       & Tests assignment to variables at different levels \\
\texttt{pass/outputspec.488}                   & Oren Watson       & Tests the output in a simple proof of concept \\
\texttt{pass/qsort.488}                        & Mike Qin          & An implementation of quicksort \\
\texttt{pass/quicksort.488}                    & Oren Watson       & Quicksort, using inner functions instead of passing variables \\
\texttt{fail/assembler\_large\_program.488}    & Daniel Bloemendal & A large program whose generated code exceeds machine memory \\
\texttt{fail/codegen\_large\_arrays.488}       & Daniel Bloemendal & Defines an array whose size exceeds machine memory \\
\texttt{fail/codegen\_large\_integer.488}      & Daniel Bloemendal & Uses an integer outside of the integer bounds of the machine \\
\texttt{fail/codegen\_large\_locals.488}       & Daniel Bloemendal & Declares a scope whose locals size exceeds available stack memory \\
\texttt{fail/codegen\_large\_string.488}       & Daniel Bloemendal & Uses a string whose length is greater than 255 characters \\
\texttt{fail/runtime\_bounds1d.488}            & Daniel Bloemendal & Out of bounds access to a 1D array (use compiler option -B s or -B e, simple or enhanced bounds checking) \\
\texttt{fail/runtime\_bounds2d.488}            & Daniel Bloemendal & Out of bounds access to a 2D array (use compiler option -B s or -B e, simple or enhanced bounds checking) \\
\texttt{fail/runtime\_deeply\_nested.488}      & Daniel Bloemendal & A program with deeply nested major scopes \\
\texttt{fail/runtime\_infinite\_recursion.488} & Daniel Bloemendal & Infinitely recursive function, leading to a stack overflow \\
\texttt{fail/runtime\_integer\_overflow.488}   & Daniel Bloemendal & Performs arithmetic operations that lead to an overflow \\
\texttt{fail/runtime\_undefined\_return.488}   & Daniel Bloemendal & Declares a function without a result statement, and attempts to use the undefined return value \\
\end{tabular}



% ----------------------------------------------------------------
\end{document}
% ----------------------------------------------------------------
